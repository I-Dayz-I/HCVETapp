\documentclass[14pt]{beamer}
\usetheme{Madrid}
\usepackage[utf8]{inputenc}
\usepackage[spanish]{babel}
\usepackage{amsmath}
\usepackage{amsfonts}
\usepackage{amssymb}
\usepackage{graphicx}
\definecolor{UBCblue}{rgb}{50, 0.1, 0.1}

\usecolortheme[named=UBCblue]{structure}

\author[Campanería, Fernández, Fuentes]
{David Campanería Cisneros\\Pablo Adrian Fuentes González\\Dayron Fernández Acosta}
\title[Aplicación   HCVet]
{HCVet: Aplicación móvil para historia clínica veterinaria.}
%\setbeamercovered{transparent} 
%\setbeamertemplate{navigation symbols}{} 
\logo{\includegraphics[height=1cm]{Images/clean_app_icon.png}} 
\institute[UH]
{\textbf{Tutores:}\\ José Alejandro Mesejo Chiong\\ José Luis Castañeda Lorenzo} 
%\date{} 
%\subject{} 
\begin{document}

\begin{frame}
\titlepage
\end{frame}



\begin{frame}
\frametitle{Temática}
\begin{block}{Temática}
Creación de una herramienta que permita gestionar los datos clínicos históricos de la condición de salud y los servicios que han recibido animales domésticos.

\end{block}

\footnote{Dayron}
\end{frame}


\begin{frame}
\frametitle{Objetivos}
El desarrollo de la app debe tener en cuenta :
\begin{itemize}
\item Facilidad de uso
\item Almacenamiento de Datos y características de animales
\item Sistema de registro de usuarios
\item Almacenamiento de Historiales Clínicos
\item Compartimiento de Datos 

\footnote{Dayron}
\end{itemize}
\end{frame}

\begin{frame}
\frametitle{Importancia del problema}
Razones por las que son necesarias resolver el problema:
\begin{itemize}
\item Gran cantidad de datos
\item Poca agilidad del sistema actual
\item Datos difícilmente transferibles
\item Inconsistencia de los Datos
\end{itemize}
\footnote{Dayron}
\end{frame}

\begin{frame}
\frametitle{Funcionalidades de la aplicación}
\begin{itemize}
\item Creación de una mascota:
\item Eliminar mascota
\item Compartir/Recibir mascota
\item Insertar nueva consulta
\item Insertar notas extras
\item Visualización
\end{itemize}
\footnote{David}
\end{frame}


\begin{frame}
\frametitle{Tecnologías utilizadas}

Tecnologías utilizadas:
\begin{itemize}
\item Flutter
\item Kotlin
\item SQLite
\end{itemize}
\footnote{David}
\end{frame}


\begin{frame}
\frametitle{Patrón Arquitectónico}

Fue utilizado un patrón \textbf{Model-View-ViewModel}.
\\
Componentes:
\begin{itemize}
\item Model
\item View
\item ViewModel
\end{itemize}

\footnote{David}
\end{frame}





\begin{frame}
\frametitle{Estructura de Model}

\begin{itemize}
\item Componente de comunicación con el servidor (Online)

\item Componente de transferencia de datos no sincronizada sin conexión a internet. (Offline)

\item Componente de almacenamiento interno (Database)

\end{itemize}

\footnote{David}
\end{frame}

\begin{frame}
\frametitle{Componente de comunicación con el servidor.}

\begin{block}{}
Se establece comunicación con el servidor a través del protocolo HTTPS.
\end{block}
\footnote{David}
\end{frame}

\begin{frame}
\frametitle{Componente de transferencia de datos offline.}

\begin{block}{}
En la implementación de este componente fue utilizado Kotlin para hacer uso del API WifiP2pManager de Android.
\end{block}
\footnote{David}
\end{frame}

\begin{frame}
\frametitle{Componente de almacenamiento interno.}

\textbf{Algunos de los métodos proporcionados por el paquete sqflite para el manejo de base de datos SQLite:}

\begin{itemize}


\item openDatabase(...)
\item getDatabasesPath()
\item execute(...)
\item insert(...)
\item update(...)
\item query(...)
\item delete(...)
\end{itemize}
\footnote{Pablo}
\end{frame}



\begin{frame}
\frametitle{Modelo de Datos}

\begin{center}

\includegraphics[scale =0.35]{Images/symplifiedClass.jpg}

\end{center}

\footnote{Pablo}
\end{frame}


\begin{frame}
\frametitle{Estructura de View}
Los 5 principios utilizados para el diseño de la interfaz:
\begin{itemize}
\item Simplicidad
\item Eficiencia
\item Consistencia
\item Retroalimentación (Feedback)
\item Accesibilidad
\end{itemize}
\footnote{Dayron}
\end{frame}

\begin{frame}
\frametitle{Estructura de View}
Las dos aproximaciones utilizadas para el diseño:

\begin{columns}
\begin{column}{0.5\textwidth}
\textbf{Menu-driven interface}
\begin{center}

\fbox{\includegraphics[scale = 0.12]{Images/homePage.jpg}}

\end{center}
\end{column}
\begin{column}{0.5\textwidth}
\textbf{Form-based interface}
\begin{center}

\fbox{\includegraphics[scale = 0.12]{Images/form.jpg}}

\end{center}
\end{column}
\end{columns}
\footnote{Dayron}
\end{frame}


\begin{frame}
\frametitle{Estructura de View-Model}
Componentes del View-Model:
\begin{itemize}
\item SyncroVM
\item KotlinChannelVM
\item DataBaseVM


\end{itemize}



\footnote{Dayron}
\end{frame}

\begin{frame}
\frametitle{Conclusiones}
\begin{itemize}
\begin{columns}
\begin{column}{0.5\textwidth}

\begin{center}

\fbox{\includegraphics[scale = 0.12]{Images/initialPage.jpg}}

\end{center}
\end{column}
\begin{column}{0.5\textwidth}

\begin{center}

\fbox{\includegraphics[scale = 0.12]{Images/pet.jpg}}

\end{center}
\end{column}
\end{columns}
\end{itemize}
\footnote{Pablo}
\end{frame}


\begin{frame}
\frametitle{Recomendaciones}

\begin{itemize}
\item Otros tipos de consultas.
\item Modificar datos.
\item Sistema de citas controlado por el servidor.
\item Diferenciar entre distintos tipos de usuarios.
\item Sistema de avisos y alarmas.
\end{itemize}

\footnote{Pablo}
\end{frame}





\begin{frame}
\maketitle
\end{frame}




\begin{frame}
\frametitle{Primera pregunta del oponente}
\begin{block}{Pregunta 1}
Aparente contradicción entre la selección de una tecnología en la hipótesis y el contenido de una de las tareas planteadas.
\end{block}
\end{frame}


\begin{frame}
\frametitle{Primera pregunta del oponente}
\begin{block}{Hipótesis}
"...sobre la plataforma Flutter a través de Dart, es posible crear un interfaz de usuario funcional..."
\end{block}

\begin{alertblock}{Tareas}
\begin{itemize}
\item ''Analizar y probar tecnologías de desarrollo de aplicaciones móviles ..."
\end{itemize}

\end{alertblock}
\end{frame}

\begin{frame}
\frametitle{Tecnologías analizadas}
\begin{itemize}
\item Java
\item Kotlin
\item Xamarin
\item React Native
\item \textbf{Flutter}

\end{itemize}

\end{frame}


\begin{frame}
\frametitle{Segunda pregunta del oponente}
\begin{block}{Pregunta 2}
¿Han considerado la idea de incluir fotos e inlcuso videos del comportamiento del animal como parte de la historia, pensando en futuras comparaciones visuales por parte de los profesionales al consultarlos?
\end{block}
\end{frame}




\begin{frame}
\frametitle{Primeros diseños de la aplicaión(7/2022)}


\begin{columns}
\begin{column}{0.5\textwidth}
\begin{center}

\includegraphics[scale = 0.35]{Images/photoExample.png}

\end{center}
\end{column}
\begin{column}{0.5\textwidth}
\begin{center}

\includegraphics[scale = 0.55]{Images/photoExampleZoom.png}

\end{center}
\end{column}
\end{columns}

\end{frame}

\end{document}
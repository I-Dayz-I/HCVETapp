\chapter*{Conclusiones}\label{chapter:conclusions}

Controlar la información medica de una mascota es un tema de suma importancia, que es llevado a cabo en su mayoría de manera poca eficiente y raramente digital. Durante el proceso investigativo que los autores emplearon en la realización de este trabajo de diploma, encontraron clara la necesidad de la existencia de un programa que permitiera el almacenamiento, organización y distribución de datos en el sector veterinario. Con el desarrollo de HCVet, se le dio cumplimiento a los objetivos trazados en el capítulo 1. La aplicación construida es capaz de mostrar información contenida en la historia clínica veterinaria de las mascotas del usuario sin necesidad de una conexión a Internet, así como añadir nueva información por medio de formularios construidos de forma dinámica a partir de distintas categorías. Siendo también posible compartir información entre usuarios utilizando la wifi. En este trabajo, se muestra el diseño de la base de datos utilizada, y algunos de los problemas surgidos durante la implementación, así como los mecanismos utilizados para dar solución a estas problemáticas. La aplicación desarrollada muestra un gran potencial que soluciona no solo los problemas planteados, sino la habilidad para expandirse a otros campos u objetivos. 